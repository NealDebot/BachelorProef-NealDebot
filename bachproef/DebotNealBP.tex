%===============================================================================
% LaTeX sjabloon voor de bachelorproef toegepaste informatica aan HOGENT
% Meer info op https://github.com/HoGentTIN/latex-hogent-report
%===============================================================================

\documentclass[dutch,dit,thesis]{hogentreport}

% TODO:
% - If necessary, replace the option `dit`' with your own department!
%   Valid entries are dbo, dbt, dgz, dit, dlo, dog, dsa, soa
% - If you write your thesis in English (remark: only possible after getting
%   explicit approval!), remove the option "dutch," or replace with "english".

\usepackage{lipsum} % For blind text, can be removed after adding actual content

%% Pictures to include in the text can be put in the graphics/ folder
\graphicspath{{../graphics/}}

%% For source code highlighting, requires pygments to be installed
%% Compile with the -shell-escape flag!
%% \usepackage[chapter]{minted}
%% If you compile with the make_thesis.{bat,sh} script, use the following
%% import instead:
\usepackage[chapter,outputdir=../output]{minted}
\usemintedstyle{solarized-light}

%% Formatting for minted environments.
\setminted{%
    autogobble,
    frame=lines,
    breaklines,
    linenos,
    tabsize=4
}

%% Ensure the list of listings is in the table of contents
\renewcommand\listoflistingscaption{%
    \IfLanguageName{dutch}{Lijst van codefragmenten}{List of listings}
}
\renewcommand\listingscaption{%
    \IfLanguageName{dutch}{Codefragment}{Listing}
}
\renewcommand*\listoflistings{%
    \cleardoublepage\phantomsection\addcontentsline{toc}{chapter}{\listoflistingscaption}%
    \listof{listing}{\listoflistingscaption}%
}

% Other packages not already included can be imported here

%%---------- Document metadata -------------------------------------------------
% TODO: Replace this with your own information
\author{Neal Debot}
\supervisor{Dhr. B. Vertonghen}
\cosupervisor{Mevr. S. Van Rampelberg}
\title%
    {Ontwikkeling en evaluatie van een proof-of-concept interace voor het efficiënt beoordelen van de toegangkelijkheid van Vlaamse huisartsenpraktijken}
\academicyear{\advance\year by -1 \the\year--\advance\year by 1 \the\year}
\examperiod{1}
\degreesought{\IfLanguageName{dutch}{Professionele bachelor in de toegepaste informatica}{Bachelor of applied computer science}}
\partialthesis{false} %% To display 'in partial fulfilment'
%\institution{Internshipcompany BVBA.}

%% Add global exceptions to the hyphenation here
\hyphenation{back-slash}

%% The bibliography (style and settings are  found in hogentthesis.cls)
\addbibresource{bachproef.bib}            %% Bibliography file
\addbibresource{../voorstel/voorstel.bib} %% Bibliography research proposal
\defbibheading{bibempty}{}

%% Prevent empty pages for right-handed chapter starts in twoside mode
\renewcommand{\cleardoublepage}{\clearpage}

\renewcommand{\arraystretch}{1.2}

%% Content starts here.
\begin{document}

%---------- Front matter -------------------------------------------------------

\frontmatter

\hypersetup{pageanchor=false} %% Disable page numbering references
%% Render a Dutch outer title page if the main language is English
\IfLanguageName{english}{%
    %% If necessary, information can be changed here
    \degreesought{Professionele Bachelor toegepaste informatica}%
    \begin{otherlanguage}{dutch}%
       \maketitle%
    \end{otherlanguage}%
}{}

%% Generates title page content
\maketitle
\hypersetup{pageanchor=true}

\input{voorwoord}
\input{samenvatting}

%---------- Inhoud, lijst figuren, ... -----------------------------------------

\tableofcontents

% In a list of figures, the complete caption will be included. To prevent this,
% ALWAYS add a short description in the caption!
%
%  \caption[short description]{elaborate description}
%
% If you do, only the short description will be used in the list of figures

\listoffigures

% If you included tables and/or source code listings, uncomment the appropriate
% lines.
\listoftables

\listoflistings

% Als je een lijst van afkortingen of termen wil toevoegen, dan hoort die
% hier thuis. Gebruik bijvoorbeeld de ``glossaries'' package.
% https://www.overleaf.com/learn/latex/Glossaries

%---------- Kern ---------------------------------------------------------------

\mainmatter{}

% De eerste hoofdstukken van een bachelorproef zijn meestal een inleiding op
% het onderwerp, literatuurstudie en verantwoording methodologie.
% Aarzel niet om een meer beschrijvende titel aan deze hoofdstukken te geven of
% om bijvoorbeeld de inleiding en/of stand van zaken over meerdere hoofdstukken
% te verspreiden!

%%=============================================================================
%% Inleiding
%%=============================================================================

\chapter{\IfLanguageName{dutch}{Inleiding}{Introduction}}%
\label{ch:inleiding}

% De inleiding moet de lezer net genoeg informatie verschaffen om het onderwerp te begrijpen en in te zien waarom de onderzoeksvraag de moeite waard is om te onderzoeken. In de inleiding ga je literatuurverwijzingen beperken, zodat de tekst vlot leesbaar blijft. Je kan de inleiding verder onderverdelen in secties als dit de tekst verduidelijkt. Zaken die aan bod kunnen komen in de inleiding~\autocite{Pollefliet2011}:

% \begin{itemize}
%   \item context, achtergrond
%   \item afbakenen van het onderwerp
%   \item verantwoording van het onderwerp, methodologie
%   \item probleemstelling
%   \item onderzoeksdoelstelling
%   \item onderzoeksvraag
%   \item \ldots
% \end{itemize}

De toegankelijkheid van huisartsenpraktijken vormt een groeiende uitdaging binnen de Vlaamse
eerstelijnszorg. Huisartsen worden vandaag geconfronteerd met een stijgende werkdruk,
toenemende administratieve lasten en een dalende instroom van nieuwe artsen in de huisartsopleiding \autocite{Rampelberg2025a}.
Om de situatie in kaart te brengen en gericht bij te sturen, maken zorgorganisaties zoals Domus Medica,
UGent en eerstelijnszones gebruik van uitgebreide indicatorensets en bevragingen. In de praktijk
worden deze evaluaties echter vaak als tijdrovend, onduidelijk en moeilijk hanteerbaar ervaren,
wat ertoe leidt dat huisartsen ze minder frequent of inconsistent invullen \autocite{Merckx2023}. Dit belemmert niet alleen
de kwaliteit van de dataverzameling, maar ook de mogelijkheid tot structurele beleidsverbetering.
Deze bachelorproef vertrekt vanuit deze concrete nood en onderzoekt in welke mate een digitale
proof-of-concept interface de toegankelijkheidsevaluatie voor huisartsen kan vereenvoudigen.

\section{\IfLanguageName{dutch}{Probleemstelling}{Problem Statement}}%
\label{sec:probleemstelling}

% Uit je probleemstelling moet duidelijk zijn dat je onderzoek een meerwaarde heeft voor een concrete doelgroep. De doelgroep moet goed gedefinieerd en afgelijnd zijn. Doelgroepen als ``bedrijven,'' ``KMO's'', systeembeheerders, enz.~zijn nog te vaag. Als je een lijstje kan maken van de personen/organisaties die een meerwaarde zullen vinden in deze bachelorproef (dit is eigenlijk je steekproefkader), dan is dat een indicatie dat de doelgroep goed gedefinieerd is. Dit kan een enkel bedrijf zijn of zelfs één persoon (je co-promotor/opdrachtgever).
Toegankelijkheid van gezondheidszorg is een fundamenteel recht voor elke burger. In Vlaanderen
heeft de overheid de taak om deze toegankelijkheid te organiseren en te bewaken, waarbij de
huisarts als centrale toegangspoort tot de gezondheidszorg een sleutelrol vervult. Toch daalt
de toegankelijkheid van huisartsenpraktijken al jaren gestaag. Een vergrijzende artsenpopulatie,
een lage instroom in de huisartsopleiding, een hoge zorgvraag bij patiënten en een groeiende
administratieve last zorgen voor een structurele druk op het systeem \autocite{Rampelberg2025}.

Om deze problematiek te monitoren en aan te pakken, werden door organisaties zoals Domus Medica
en UGent indicatorensets en vragenlijsten ontwikkeld die huisartsen en praktijkmedewerkers
toelaten de toegankelijkheid van hun praktijk systematisch te evalueren \autocite{Merckx2023}. Uit pilootstudies en
gesprekken met huisartsen binnen Eerstelijnszone Aalst blijkt echter dat deze evaluatie-instrumenten
in de praktijk ernstige tekortkomingen vertonen. Huisartsen geven aan dat de bestaande vragenlijsten
te lang en te complex zijn om naast de consultaties in te vullen. Bovendien laten de criteria ruimte
voor uiteenlopende interpretaties, waardoor de betrouwbaarheid en vergelijkbaarheid van de
verzamelde data sterk vermindert \autocite{Rampelberg2025a}. Bijgevolg worden de evaluaties slechts sporadisch en inconsistent
uitgevoerd, wat de bruikbaarheid voor beleidsvorming ernstig beperkt.

Er bestaat momenteel geen gevalideerd digitaal hulpmiddel dat specifiek afgestemd is op de
toegankelijkheidsevaluaties van Vlaamse huisartsenpraktijken. De combinatie van tijdsdruk,
interpretatieproblemen en het gebrek aan een gebruiksvriendelijke tool vormt dan ook de kern
van het probleem dat deze bachelorproef wil aanpakken.


\section{\IfLanguageName{dutch}{Onderzoeksvraag}{Research question}}%
\label{sec:onderzoeksvraag}

% Wees zo concreet mogelijk bij het formuleren van je onderzoeksvraag. Een onderzoeksvraag is trouwens iets waar nog niemand op dit moment een antwoord heeft (voor zover je kan nagaan). Het opzoeken van bestaande informatie (bv. ``welke tools bestaan er voor deze toepassing?'') is dus geen onderzoeksvraag. Je kan de onderzoeksvraag verder specifiëren in deelvragen. Bv.~als je onderzoek gaat over performantiemetingen, dan 

De centrale onderzoeksvraag van dit onderzoek luidt:

\begin{quote}
  \textit{In welke mate kan een tested proof-of-concept interface bestaande
  toegankelijkheidsevaluaties voor huisartsen en praktijkmedewerkers sneller, consistenter
  en gebruiksvriendelijker maken dan de huidige vragenlijsten?}
\end{quote}

Om deze hoofdvraag concreet en onderbouwd te beantwoorden, worden twee aanvullende deelvragen
geformuleerd. Ten eerste wordt nagegaan in welke mate de interface bijdraagt aan een meer
consistente interpretatie en beantwoording van toegankelijkheidsindicatoren door huisartsen.
Ten tweede wordt onderzocht in welke mate de interface de cognitieve belasting vermindert tijdens
het uitvoeren van een toegankelijkheidsevaluatie in vergelijking met een klassieke vragenlijst.
Samen maken deze deelvragen het mogelijk om niet alleen vast te stellen óf de interface een
verbetering vormt, maar ook waaróm en op welke manier zij effectiever is dan de huidige aanpak.

\section{\IfLanguageName{dutch}{Onderzoeksdoelstelling}{Research objective}}%
\label{sec:onderzoeksdoelstelling}

% Wat is het beoogde resultaat van je bachelorproef? Wat zijn de criteria voor succes? Beschrijf die zo concreet mogelijk. Gaat het bv.\ om een proof-of-concept, een prototype, een verslag met aanbevelingen, een vergelijkende studie, enz.

Het doel van dit toegepaste onderzoek is tweeledig. Enerzijds wordt onderzocht welke elementen
van de huidige evaluatie-instrumenten het meeste tijdverlies, verwarring of inconsistentie
veroorzaken bij huisartsen en praktijkmedewerkers. Anderzijds wordt op basis van deze inzichten
een proof-of-concept digitale interface ontworpen en empirisch geëvalueerd.

Het beoogde eindresultaat is een werkend prototype dat de toegankelijkheidsevaluatie structureert,
visueel ondersteunt en efficiënter maakt, aangevuld met een analyse van de gebruikservaring. De
bachelorproef kan als succesvol beschouwd worden wanneer de interface aantoont dat zij de invultijd
reduceert, de gebruiksvriendelijkheid verhoogt en het evaluatieproces voor huisartsen en
praktijkmedewerkers daadwerkelijk vereenvoudigt. Daartoe wordt een combinatie van kwantitatieve
en kwalitatieve evaluatiemethoden ingezet: tijdsmetingen, de System Usability Scale (SUS),
de think-aloud-methode en heatmap-analyses via tools zoals Microsoft Clarity.

Hoewel toegankelijkheid een breed en multidimensionaal begrip is, focust deze bachelorproef
bewust op een afgebakende selectie van toegankelijkheidsindicatoren, namelijk die indicatoren
die op basis van literatuur en verkennende interviews het vaakst als tijdrovend, moeilijk
interpreteerbaar of inconsistent worden ervaren. Door deze afbakening kan de effectiviteit van
de proof-of-concept interface op een haalbare en diepgaande manier worden onderzocht binnen
de context van Eerstelijnszone Aalst.

\section{\IfLanguageName{dutch}{Opzet van deze bachelorproef}{Structure of this bachelor thesis}}%
\label{sec:opzet-bachelorproef}

% Het is gebruikelijk aan het einde van de inleiding een overzicht te
% geven van de opbouw van de rest van de tekst. Deze sectie bevat al een aanzet
% die je kan aanvullen/aanpassen in functie van je eigen tekst.

De rest van deze bachelorproef is als volgt opgebouwd:

In Hoofdstuk~\ref{ch:stand-van-zaken} wordt een overzicht gegeven van de stand van zaken binnen het onderzoeksdomein, op basis van een literatuurstudie.

In Hoofdstuk~\ref{ch:methodologie} wordt de methodologie toegelicht en worden de gebruikte onderzoekstechnieken besproken om een antwoord te kunnen formuleren op de onderzoeksvragen.

% TODO: Vul hier aan voor je eigen hoofstukken, één of twee zinnen per hoofdstuk

In Hoofdstuk~\ref{ch:conclusie}, tenslotte, wordt de conclusie gegeven en een antwoord geformuleerd op de onderzoeksvragen. Daarbij wordt ook een aanzet gegeven voor toekomstig onderzoek binnen dit domein.
\chapter{\IfLanguageName{dutch}{Stand van zaken}{State of the art}}%
\label{ch:stand-van-zaken}

% Tip: Begin elk hoofdstuk met een paragraaf inleiding die beschrijft hoe
% dit hoofdstuk past binnen het geheel van de bachelorproef. Geef in het
% bijzonder aan wat de link is met het vorige en volgende hoofdstuk.

% Pas na deze inleidende paragraaf komt de eerste sectiehoofding.

% Dit hoofdstuk bevat je literatuurstudie. De inhoud gaat verder op de inleiding, maar zal het onderwerp van de bachelorproef *diepgaand* uitspitten. De bedoeling is dat de lezer na lezing van dit hoofdstuk helemaal op de hoogte is van de huidige stand van zaken (state-of-the-art) in het onderzoeksdomein. Iemand die niet vertrouwd is met het onderwerp, weet nu voldoende om de rest van het verhaal te kunnen volgen, zonder dat die er nog andere informatie moet over opzoeken \autocite{Pollefliet2011}.

% Je verwijst bij elke bewering die je doet, vakterm die je introduceert, enz.\ naar je bronnen. In \LaTeX{} kan dat met het commando \texttt{$\backslash${textcite\{\}}} of \texttt{$\backslash${autocite\{\}}}. Als argument van het commando geef je de ``sleutel'' van een ``record'' in een bibliografische databank in het Bib\LaTeX{}-formaat (een tekstbestand). Als je expliciet naar de auteur verwijst in de zin (narratieve referentie), gebruik je \texttt{$\backslash${}textcite\{\}}. Soms is de auteursnaam niet expliciet een onderdeel van de zin, dan gebruik je \texttt{$\backslash${}autocite\{\}} (referentie tussen haakjes). Dit gebruik je bv.~bij een citaat, of om in het bijschrift van een overgenomen afbeelding, broncode, tabel, enz. te verwijzen naar de bron. In de volgende paragraaf een voorbeeld van elk.

% \textcite{Knuth1998} schreef een van de standaardwerken over sorteer- en zoekalgoritmen. Experten zijn het erover eens dat cloud computing een interessante opportuniteit vormen, zowel voor gebruikers als voor dienstverleners op vlak van informatietechnologie~\autocite{Creeger2009}.

% Let er ook op: het \texttt{cite}-commando voor de punt, dus binnen de zin. Je verwijst meteen naar een bron in de eerste zin die erop gebaseerd is, dus niet pas op het einde van een paragraaf.

% \begin{figure}
%   \centering
%   \includegraphics[width=0.8\textwidth]{grail.jpg}
%   \caption[Voorbeeld figuur.]{\label{fig:grail}Voorbeeld van invoegen van een figuur. Zorg altijd voor een uitgebreid bijschrift dat de figuur volledig beschrijft zonder in de tekst te moeten gaan zoeken. Vergeet ook je bronvermelding niet!}
% \end{figure}

% \begin{listing}
%   \begin{minted}{python}
%     import pandas as pd
%     import seaborn as sns

%     penguins = sns.load_dataset('penguins')
%     sns.relplot(data=penguins, x="flipper_length_mm", y="bill_length_mm", hue="species")
%   \end{minted}
%   \caption[Voorbeeld codefragment]{Voorbeeld van het invoegen van een codefragment.}
% \end{listing}

% \lipsum[7-20]

% \begin{table}
%   \centering
%   \begin{tabular}{lcr}
%     \toprule
%     \textbf{Kolom 1} & \textbf{Kolom 2} & \textbf{Kolom 3} \\
%     $\alpha$         & $\beta$          & $\gamma$         \\
%     \midrule
%     A                & 10.230           & a                \\
%     B                & 45.678           & b                \\
%     C                & 99.987           & c                \\
%     \bottomrule
%   \end{tabular}
%   \caption[Voorbeeld tabel]{\label{tab:example}Voorbeeld van een tabel.}
% \end{table}

In dit hoofdstuk wordt de bestaande kennis en literatuur rond het onderwerp van deze
bachelorproef systematisch in kaart gebracht. De stand van zaken is opgebouwd rond drie
grote thema's. Eerst wordt het probleemdomein geschetst: wat is de context van
toegankelijkheidsevaluaties in de Vlaamse huisartsenpraktijk, welke studies bestaan er al,
waar knelt het schoentje en voor wie wordt dit onderzoek gevoerd? Vervolgens wordt het
technologisch landschap verkend: welke tools bestaan er vandaag voor digitale bevragingen,
hoe verhouden die zich tot de noden van dit project, en welke keuzes zijn gemaakt voor het
framework en de authenticatie? Ten slotte worden de succescriteria en evaluatiemethoden
besproken die bepalen of de proof-of-concept interface haar doel bereikt.

\section{Het probleemdomein}
\label{sec:probleemdomein}

\subsection{Bestaande studies}
\label{subsec:bestaande-studies}

De toegankelijkheid van huisartsenpraktijken is in Vlaanderen de afgelopen jaren een
toenemend beleidsprioriteit geworden. Het Federaal Kenniscentrum voor de Gezondheidszorg
(KCE) onderwerpt het Belgische gezondheidszorgsysteem regelmatig aan een grondige analyse,
waarbij toegankelijkheid van zorg een van de beoordelingscriteria is \autocite{Merckx2023}.
Ondanks de brede erkenning van het probleem zijn er in Vlaanderen slechts een beperkt aantal
wetenschappelijke studies die de toegankelijkheid van de huisartsenpraktijk systematisch en
empirisch onderzoeken \autocite{Rampelberg2025}.\\

Het meest omvangrijke Vlaamse onderzoek op dit vlak is het rapport Toegankelijkheid
Huisartsgeneeskunde, uitgevoerd in opdracht van Domus Medica, UGent (Vakgroep
Volksgezondheid en Eerstelijnszorg), AContrario en VIVEL \autocite{Merckx2023}. Dit project
doorliep twee pilootfases en ontwikkelde stapsgewijs een gevalideerde indicatorenset voor het
meten van toegankelijkheid. Vertrekkend vanuit een literatuuronderzoek werd een initiële set
van 129 indicatoren opgesteld (set 1.0), die op basis van relevantie, beschikbaarheid en
haalbaarheid werd gereduceerd tot 58 (set 2.0). Na cognitieve interviews en analyse van de
verzamelde pilootdata werd de set verder verfijnd tot 51 indicatoren (set 3.0). Een
interbeoordelaarsanalyse, test-hertestsrudie en expertenbevraging resulteerden uiteindelijk in
een finale set 4.0 van 38 indicatoren, die zowel zorgaanbod als zorgnood omvat.\\

In Nederland beschikt het onderzoekscentrum NIVEL al jarenlang over een actief
monitoringsysteem voor huisartsenpraktijken, dat grootschalige en longitudinale data oplevert
over organisatie, capaciteit en toegankelijkheid van de eerstelijnszorg \autocite{Rampelberg2025}.
Dit contrasteert sterk met de Vlaamse situatie, waar gestructureerde en herhaalde metingen
vooralsnog ontbreken. De zoektocht naar relevant materiaal voor de Vlaamse context vereist
dan ook een uitbreiding naar vakbladen, nieuwsberichten en grijze literatuur.

\subsection{Pijnpunten in de huidige evaluatie-instrumenten}
\label{subsec:pijnpunten}

Ondanks de wetenschappelijke kwaliteit van de ontwikkelde indicatorenset en vragenlijsten,
toont het rapport Toegankelijkheid Huisartsgeneeskunde aan dat de praktische inzetbaarheid
ervan ernstig belemmerd wordt door een reeks structurele knelpunten \autocite{Merckx2023}.\\

Een eerste en meest fundamenteel pijnpunt is de \textbf{lengte en complexiteit} van de bevraging.
Huisartsenkringen geven aan dat de omvang van de vragenlijst en de uitgebreide set criteria
een bijkomende werklast betekenen voor huisartsen die toch al onder tijdsdruk staan. De
verwachting is dan ook realistisch dat artsen geen tijd zullen nemen om de vragenlijst volledig
en correct in te vullen \autocite{Merckx2023}.\\

Een tweede knelpunt is de \textbf{inconsistente interpretatie} van indicatoren. Uit de
interbeoordelaarsanalyse bleek dat meerdere categorische variabelen een beperkte
betrouwbaarheid vertonen wanneer twee medewerkers uit dezelfde praktijk de vragenlijst
onafhankelijk invullen. Dit wijst erop dat de formulering van bepaalde vragen voor
uiteenlopende interpretaties vatbaar is, wat de validiteit van de verzamelde data
ondermijnt \autocite{Merckx2023}. De test-herteststudie bevestigde bovendien dat antwoorden
soms ongewild beïnvloed worden door omstandigheden zoals het tijdstip van invullen of de
lengte van de bevraging.\\

Een derde pijnpunt is het \textbf{wantrouwen} ten aanzien van de dataverzameling en het gebruik
ervan. Kringen vrezen dat de overheid de verzamelde data zal gebruiken als controlemiddel,
bijvoorbeeld door praktijken te vergelijken op harde outputcijfers zoals het aantal
patiëntenconsulten, zonder rekening te houden met de specifieke zorglast of context van een
praktijk \autocite{Merckx2023}. Dit wantrouwen leidt ertoe dat kringen terughoudend zijn om
deel te nemen, wat de representativiteit van de data beperkt.\\

Daarnaast werd vastgesteld dat bepaalde indicatoren in hun oorspronkelijke formulering niet
\textbf{relevant} bleken of geen variatie vertoonden in de antwoorden, waardoor zij weinig
onderscheidende waarde hadden. Zo bleek de vraag over de aanwezigheid van een chauffeur
voor wachtdiensten steeds hetzelfde antwoord op te leveren, en werd de werkdrukschaal in
zijn oorspronkelijke vorm als voor interpretatie vatbaar beschouwd en vervangen door vier
nieuwe gevalideerde subschalen \autocite{Merckx2023}.\\

Het rapport formuleert hierover een expliciete aanbeveling: de belasting bij het werkveld moet
geminimaliseerd worden via strategieën zoals een doordachte selectie van indicatoren, reliëf
in de bevragingsfrequentie en slimme automatisering, zoals het automatisch invullen van
velden op basis van bestaande databronnen (IMA, mutualiteiten) \autocite{Merckx2023}. Deze
aanbeveling vormt een directe aanleiding voor het onderzoek in deze bachelorproef.

\subsection{Doelgroep}
\label{subsec:doelgroep}

De primaire doelgroep van deze bachelorproef bestaat uit \textbf{huisartsen en
praktijkmedewerkers} werkzaam binnen Eerstelijnszone Aalst. Binnen de Vlaamse
eerstelijnszorg zijn grote verschillen zichtbaar in praktijkwerking, schaalgrootte en digitale
maturiteit: sommige praktijken werken volledig solo, andere zijn uitgegroeid tot grote
groepspraktijken met verpleegkundigen, administratief personeel en telefonische
secretariaatsdiensten \autocite{Rampelberg2025a}. Deze diversiteit maakt het bijzonder
uitdagend om een uniforme evaluatietool te ontwerpen die voor alle praktijkvormen even
bruikbaar is.\\

Huisartsen beschikken over zeer beperkte tijd buiten de consultaties voor administratieve
taken. Elke bijkomende belasting, zoals het invullen van een uitgebreide vragenlijst, concurreert
rechtstreeks met patiëntenzorg \autocite{Merckx2023}. Praktijkmedewerkers, zoals
verpleegkundigen of administratief personeel, worden soms ingezet voor het invullen van de
praktijkvragenlijst, maar ook zij beschikken over beperkte beschikbaarheid en hebben vaak
geen gespecialiseerde kennis van alle gevraagde indicatoren.\\

De effectiviteitsladder die werd opgesteld binnen het project van Eerstelijnszone Aalst
illustreert dat de noden en mogelijkheden sterk verschillen naargelang de praktijkvorm: een
solopraktijk die overweegt een verpleegkundige aan te werven heeft andere evaluatienoden
dan een groepspraktijk die haar online agendabeheer wil optimaliseren \autocite{Rampelberg2025a}.
Een digitale interface moet dan ook flexibel genoeg zijn om met deze variatie om te gaan.

\subsection{Doel van de bachelorproef}
\label{subsec:doel-bp}

Het doel van deze bachelorproef kadert expliciet binnen de aanbevelingen uit het rapport
Toegankelijkheid Huisartsgeneeskunde en de noden die geïdentificeerd werden binnen
Eerstelijnszone Aalst. Er bestaat momenteel geen gevalideerd digitaal hulpmiddel dat
specifiek afgestemd is op de toegankelijkheidsevaluaties van Vlaamse
huisartsenpraktijken \autocite{Merckx2023}. Deze bachelorproef wil hierop een antwoord
bieden door een proof-of-concept interface te ontwikkelen die de bestaande indicatorenset
omzet in een gebruiksvriendelijke, interactieve bevraging.\\

De interface beoogt drie concrete verbeteringen ten opzichte van de huidige papieren of
digitale vragenlijsten: een kortere invultijd, een hogere consistentie in de interpretatie van
indicatoren, en een lagere cognitieve belasting voor de gebruiker \autocite{Merckx2023}.
Daarmee sluit het onderzoek aan bij de bredere beleidsdoelstelling om de frequentie en
kwaliteit van toegankelijkheidsevaluaties in Vlaamse huisartsenpraktijken structureel te
verhogen.

\section{Technologisch onderzoek}
\label{sec:technologisch-onderzoek}

\subsection{Bestaande survey-tools}
\label{subsec:survey-tools}

Voor de ontwikkeling van de proof-of-concept interface is het relevant om eerst het bestaande
landschap van digitale survey-tools te verkennen. Verschillende platformen bieden
out-of-the-box functionaliteit voor het afnemen van vragenlijsten, elk met eigen sterktes en
beperkingen in de context van een toegankelijkheidsevaluatie voor huisartsenpraktijken.\\

\textbf{Microsoft Forms} is een laagdrempelig en breed verspreid platform dat sterk geïntegreerd
is in de Microsoft 365-omgeving. Het biedt basisondersteuning voor meerdere vraagformats, conditionele logica en automatische
rapportage via Power BI. De drempel voor eindgebruikers is laag, maar de aanpassingsmogelijkheden zijn beperkt: visuele ondersteuning, automatische feedback per indicator en
aangepaste navigatielogica zijn niet of slechts beperkt realiseerbaar.\\

\textbf{REDCap} (Research Electronic Data Capture) is een veilig, webgebaseerd softwareplatform dat is ontworpen om gegevensverzameling voor onderzoek te ondersteunen.
Het biedt een intuïtieve interface voor gevalideerde gegevensverzameling, 
audittrails voor het volgen van gegevensmanipulatie en exportprocedures, 
geautomatiseerde exportprocedures voor naadloze gegevensdownloads naar gangbare statistische pakketten en procedures voor gegevensintegratie en interoperabiliteit met externe bronnen \autocite{Harris2009}.\\

\textbf{SurvaySparrow} is een online enquête- en feedbackplatform waarmee organisaties eenvoudig enquêtes kunnen opstellen, verspreiden en analyseren.
Het platform onderscheidt zich door zijn conversatiestijl, waarbij enquêtes worden gepresenteerd in een vorm die voor respondenten natuurlijker en interactiever aanvoelt.
Enquêtes kunnen via verschillende kanalen worden gedeeld, zoals e-mail, websites en sociale media, en kunnen bovendien worden geautomatiseerd voor terugkerende feedbackmomenten.
Daarnaast biedt SurveySparrow uitgebreide rapportages en realtime dashboards, zodat resultaten overzichtelijk worden weergegeven en direct inzicht geven.
Door integraties met onder andere CRM- en HR-systemen is het platform breed inzetbaar, bijvoorbeeld voor klanttevredenheidsonderzoeken, medewerkerstevredenheid en marktonderzoek.\\

\subsection{Fitgap-analyse van de bestaande tools}
\label{subsec:fitgap}

Een fitgap-analyse vergelijkt de functionele vereisten van de beoogde interface met wat de
bestaande tools kunnen bieden. Op basis van de noden die geïdentificeerd werden in de
literatuur en de gesprekken met huisartsen binnen Eerstelijnszone Aalst kunnen de volgende
vereisten worden geformuleerd \autocite{Merckx2023}:

\begin{enumerate}
  \item \textbf{Dynamische vragen}: is er mogenlijkheid voor een vragenlijst die verandert afhangend van voorgaande antwoorden?
  \item \textbf{distributie mogelijkheden}: kan de tool geïntigreerd worden in een applicatie via embedding, een api of webhooks?
  \item \textbf{Automatische invulling}: kunnen vragen automatisch worden ingevuld in de form met gekende data?
  \item \textbf{Prijzen}: wat kost het gebruiken van de tool?
\end{enumerate}

\subsubsection{Microsoft Forms}
\label{subsubsec:MicrosoftForms}

\textbf{Dynamische vragen}: Beperkte mogelijkheid. 
Je kan "branching logic" instellen waardoor respondenten naar een andere sectie worden doorgestuurd op basis van een antwoord, 
maar de opties zijn vrij eenvoudig.\\
\textbf{Distributie}: Geen officiële embedding via API of webhooks. 
Formulieren kunnen gedeeld worden via link, QR-code of ingebed als iframe in een webpagina. 
Integratie met andere tools verloopt via Power Automate (niet via native webhooks/API).\\
\textbf{Automatische invulling}: Niet standaard ondersteund. 
Wel mogelijk om bepaalde velden voor te invullen via aangepaste URL-parameters, 
maar dit is beperkt en niet officieel gedocumenteerd.\\
\textbf{Prijzen}: Gratis inbegrepen bij Microsoft 365. 
Standalone gratis versie beschikbaar met beperkingen (max. 100 antwoorden per formulier).

\subsubsection{REDCap}
\label{subsubsec:REDCap}

\textbf{Dynamische vragen}:  Uitstekend. 
REDCap heeft krachtige "branching logic" en "piping" waarmee vragen volledig dynamisch worden op basis van eerdere antwoorden.\\
\textbf{Distributie}: Sterke mogelijkheden. 
Beschikt over een API, kan ingebed worden in externe applicaties, en ondersteunt geautomatiseerde workflows. 
Geen native webhooks, maar API-integraties zijn uitgebreid.\\
\textbf{Automatische invulling}: Ja. Via de API kunnen gekende data automatisch in formulieren geladen worden. 
Ook "pre-filling" via unieke links is mogelijk.\\
\textbf{Prijzen}: Gratis voor non-profit en academische instellingen via een consortiumlicentie. 
Commercieel gebruik vereist een betalende licentie (prijs varieert per instelling/organisatie).

\subsubsection{SurvaySparrow}
\label{subsubsec:SurvaySparrow}

\textbf{Dynamische vragen}: Ja. 
Uitgebreide conditionele logica waarmee de vragenlijst volledig aanpast op basis van eerdere antwoorden. 
Ondersteunt ook "skip logic" en gepersonaliseerde flows.\\
\textbf{Distributie}: Uitstekend. 
Biedt embedding (iframe & SDK), een REST API en webhooks aan. 
Goed geschikt voor integratie in externe applicaties.\\
\textbf{Automatische invulling}: Ja. 
Via API of URL-parameters kunnen gekende data vooraf ingevuld worden in formulieren ("pre-fill" functionaliteit).\\
\textbf{Prijzen}: Betalend, met een gratis proefperiode. 
Plannen starten vanaf ongeveer €19/maand (Basic) tot €149+/maand (Business), afhankelijk van het aantal antwoorden en functies. 
Enterprise op aanvraag.

\subsubsection{Samenvatting}
\label{subsubsec:Samenvatting}

\begin{table}[h]
  \centering
  \begin{tabular}{|l|l|l|l|}
    \hline
    \textbf{Criterium} & \textbf{Microsoft Forms} & \textbf{REDCap} & \textbf{SurveySparrow} \\
    \hline
    Dynamische vragen & Beperkt & Uitstekend & Uitstekend \\
    \hline
    Distributie/API & Beperkt & Goed & Uitstekend \\
    \hline
    Auto-invulling & Beperkt & Ja & Ja \\
    \hline
    Prijs & Gratis (M365) & Gratis (non-profit) & Betaald \\
    \hline
  \end{tabular}
  \label{tab:fit-analyse}
\end{table}

\subsection{Framework}
\label{subsec:framework}

Voor de ontwikkeling van de proof-of-concept interface wordt gekozen voor \textbf{Angular},
een open-source frontend-framework ontwikkeld en onderhouden door Google. Deze keuze is
om meerdere redenen gemotiveerd.\\

Ten eerste biedt Angular een duidelijke en gestructureerde architectuur op basis van
componenten, modules en services, wat de onderhoudbaarheid en uitbreidbaarheid van de
codebase ten goede komt. Ten tweede beschikt Angular over sterke ingebouwde
ondersteuning voor formulieren en validatiemechanismen via zowel template-driven als
reactive forms, wat bijzonder relevant is voor het ontwikkelen van interactieve vragenlijsten
met conditionele logica. Ten derde laat de bestaande voorkennis en praktijkervaring met het
framework toe om de ontwikkeltijd te beperken en de focus te leggen op het onderzoeksdoel,
met name het ontwerpen en evalueren van een gebruiksvriendelijke interface.\\

Angular maakt gebruik van TypeScript, wat zorgt voor sterkere typering en vroegere
foutdetectie in vergelijking met puur JavaScript. De Angular CLI faciliteert de opzet en het
beheer van het project, inclusief het genereren van componenten, services en testbestanden.

\section{Onderzoek en succescriteria}
\label{sec:succescriteria}

\subsection{De System Usability Scale (SUS)}
\label{subsec:sus}

De System Usability Scale (SUS) is een gestandaardiseerde vragenlijst voor het meten van
de subjectief ervaren gebruiksvriendelijkheid van een systeem of interface. De schaal werd
ontwikkeld door John Brooke in 1986 en bestaat uit tien stellingen die de gebruiker beoordeelt
op een vijfpunts Likert-schaal. De berekende SUS-score situeert zich tussen 0 en 100 en laat
toe de interface te positioneren ten opzichte van internationale usabilitynormen.\\

Een SUS-score van 68 wordt beschouwd als het gemiddelde van commerciële systemen. Scores
boven 75 worden internationaal als \textit{good usability} beschouwd, terwijl scores boven 85
als \textit{excellent} worden aangemerkt. De SUS heeft het voordeel
dat het een kleine steekproef vereist (5 tot 10 gebruikers volstaan voor betrouwbare
resultaten), wat haalbaar is binnen de context van dit onderzoek.\\

In het kader van deze bachelorproef wordt een SUS-score van minimaal 75 als succescriterium
gehanteerd. Dit impliceert dat de interface door huisartsen en praktijkmedewerkers als
duidelijk gebruiksvriendelijker wordt beoordeeld dan de klassieke vragenlijst, die op basis van
de literatuur als complex en tijdrovend wordt ervaren \autocite{Merckx2023}.

\subsection{Heatmap-analyse}
\label{subsec:heatmap}

Naast de SUS-vragenlijst wordt de interactie van gebruikers met de interface gemonitord via
heatmaps en klikgegevens. Voor dit doel wordt gebruikgemaakt van tools zoals
\textbf{Microsoft Clarity} of \textbf{Hotjar}. Deze tools registreren automatisch het klikgedrag,
scrollgedrag en time-on-task van gebruikers, zonder dat hiervoor aanvullende
logging-infrastructuur moet worden opgezet.\\

Heatmaps visualiseren welke zones van de interface de meeste aandacht trekken en waar
gebruikers onverwacht gedrag vertonen, zoals herhaalde klikken op een niet-klikbaar element
(rage clicks) of het verlaten van een pagina zonder de sectie te voltooien. Dergelijke patronen
zijn een indicator voor verwarring of cognitieve belasting op specifieke plaatsen in de
interface.\\

Een succesvolle interface wordt gekenmerkt door een vloeiend navigatiepatroon zonder
significante zones van verwarring, en door een time-on-task die merkbaar lager ligt dan bij de
klassieke vragenlijst. De heatmapdata worden gebruikt als aanvulling op de kwantitatieve
tijdsmetingen en de kwalitatieve think-aloud-observaties.

\subsection{Ideale lengte van een survey}
\label{subsec:survey-lengte}

Een belangrijk inzicht uit de usability-literatuur is dat de lengte van een vragenlijst een
directe invloed heeft op de voltooiingsgraad, de kwaliteit van de antwoorden en de
cognitieve belasting van de invuller \autocite{Kost2018}. Langere surveys leiden tot
survey fatigue, waarbij respondenten minder aandachtig antwoorden of de vragenlijst
voortijdig verlaten.\\

Het rapport Toegankelijkheid Huisartsgeneeskunde erkent dit expliciet als een knelpunt:
de uiteindelijke indicatorenset (set 4.0, 38 indicatoren) is al het resultaat van meerdere
reductiestappen om de belasting te minimaliseren, maar de praktijk toont aan dat zelfs deze
gereduceerde set nog als te omvangrijk wordt ervaren \autocite{Merckx2023}. De aanbeveling
is dan ook om de bevraging te structureren in modules, zodat een praktijk niet in één sessie
alles hoeft in te vullen, en om automatische invulling van velden te voorzien op basis van
bestaande databronnen.\\

Voor de proof-of-concept interface wordt daarom gekozen voor een modulaire opbouw en
een gefaseerde presentatie van vragen. Elke module bevat een beperkt aantal indicatoren die
thematisch samenhangen. Een voortgangsindicator toont de gebruiker hoever hij of zij staat,
wat de motivatie om door te gaan verhoogt. Internationaal onderzoek suggereert dat een
survey die in tien tot maximaal 20 minuten kan worden ingevuld de beste resultaten heeft \autocite{Revilla2017}.

\subsection{Criteria voor een geslaagd onderzoek}
\label{subsec:slagen-onderzoek}

De bachelorproef wordt als succesvol beschouwd wanneer aan de volgende criteria is voldaan.
Ten eerste dient de invultijd van de vragenlijst tussen de tien tot maximaal twintig minuten, gemeten via tijdsmetingen bij een groep van een tiental huisartsen en praktijkmedewerkers. 
Ten tweede dient de SUS-score van de interface boven de drempel van 75 te liggen, 
wat internationaal geldt als \textit{good usability}.\\

Aanvullend worden de heatmapdata en de think-aloud-observaties gebruikt om specifieke
knelpunten in de interface te identificeren en te verklaren waarom bepaalde ontwerpkeuzes al
dan niet bijdragen aan de beoogde verbeteringen.

\input{methodologie}

% Voeg hier je eigen hoofdstukken toe die de ``corpus'' van je bachelorproef
% vormen. De structuur en titels hangen af van je eigen onderzoek. Je kan bv.
% elke fase in je onderzoek in een apart hoofdstuk bespreken.

%\input{...}
%\input{...}
%...

\input{conclusie}

%---------- Bijlagen -----------------------------------------------------------

\appendix

\chapter{Onderzoeksvoorstel}

Het onderwerp van deze bachelorproef is gebaseerd op een onderzoeksvoorstel dat vooraf werd beoordeeld door de promotor. Dat voorstel is opgenomen in deze bijlage.

%% TODO: 
\section*{Samenvatting}

% Kopieer en plak hier de samenvatting (abstract) van je onderzoeksvoorstel.

  De toegankelijkheid van huisartsenpraktijken wordt in Vlaanderen regelmatig geëvalueerd via indicatorensets en vragenlijsten ontwikkeld door o.a. Domus Medica en UGent.
  Uit eerdere pilootstudies blijkt dat deze evaluaties door huisartsen vaak als tijdrovend, complex en inconsistent worden ervaren, wat de frequentie en kwaliteit van zelfevaluaties beperkt.
  Dit onderzoek zal zich toespitsen op de toegankelijkheid met betrekking tot de fysieke en organisatorische aspecten van de praktijk.
  En zal onderzoeken in welke mate een proof-of-concept interface dit proces efficiënter en gebruiksvriendelijker kan maken.\\\\
  Via interviews met huisartsen wordt eerst in kaart gebracht welke elementen van de bestaande evaluaties de grootste tijdsdruk of verwarring veroorzaken.
  Op basis hiervan wordt een interactieve interface ontwikkeld die de toegankelijkheidsbevraging structureert, visuele ondersteuning biedt en automatische feedback genereert.
  De interface wordt vervolgens getest bij huisartsen aan de hand van tijdsmetingen, usability-observaties, Microsoft Clarity-heatmaps en de System Usability Scale (SUS).\\\\
  De resultaten geven inzicht in de mate waarin de interface het invulproces versnelt, de consistentie van antwoorden verhoogt, sneller feedback geeft en de gebruikerservaring verbetert.
  Het onderzoek toont aan hoe digitale ondersteuning het evalueren van toegankelijkheid kan vereenvoudigen en vormt een basis voor verdere implementatie binnen de eerstelijnszorg.

% Verwijzing naar het bestand met de inhoud van het onderzoeksvoorstel
%---------- Inleiding ---------------------------------------------------------

% TODO: Is dit voorstel gebaseerd op een paper van Research Methods die je
% vorig jaar hebt ingediend? Heb je daarbij eventueel samengewerkt met een
% andere student?
% Zo ja, haal dan de tekst hieronder uit commentaar en pas aan.

%\paragraph{Opmerking}

% Dit voorstel is gebaseerd op het onderzoeksvoorstel dat werd geschreven in het
% kader van het vak Research Methods dat ik (vorig/dit) academiejaar heb
% uitgewerkt (met medesturent VOORNAAM NAAM als mede-auteur).
% 

\section{Inleiding}%
\label{sec:inleiding}

% Waarover zal je bachelorproef gaan? Introduceer het thema en zorg dat volgende zaken zeker duidelijk aanwezig zijn:

% \begin{itemize}
%   \item kaderen thema
%   \item de doelgroep
%   \item de probleemstelling en (centrale) onderzoeksvraag
%   \item de onderzoeksdoelstelling
% \end{itemize}

% Denk er aan: een typische bachelorproef is \textit{toegepast onderzoek}, wat betekent dat je start vanuit een concrete probleemsituatie in bedrijfscontext, een \textbf{casus}. Het is belangrijk om je onderwerp goed af te bakenen: je gaat voor die \textit{ene specifieke probleemsituatie} op zoek naar een goede oplossing, op basis van de huidige kennis in het vakgebied.

% De doelgroep moet ook concreet en duidelijk zijn, dus geen algemene of vaag gedefinieerde groepen zoals \emph{bedrijven}, \emph{developers}, \emph{Vlamingen}, enz. Je richt je in elk geval op it-professionals, een bachelorproef is geen populariserende tekst. Eén specifiek bedrijf (die te maken hebben met een concrete probleemsituatie) is dus beter dan \emph{bedrijven} in het algemeen.

% Formuleer duidelijk de onderzoeksvraag! De begeleiders lezen nog steeds te veel voorstellen waarin we geen onderzoeksvraag terugvinden.

% Schrijf ook iets over de doelstelling. Wat zie je als het concrete eindresultaat van je onderzoek, naast de uitgeschreven scriptie? Is het een proof-of-concept, een rapport met aanbevelingen, \ldots Met welk eindresultaat kan je je bachelorproef als een succes beschouwen?
De toegankelijkheid van huisartsenpraktijken vormt een groeiende uitdaging binnen de Vlaamse eerstelijnszorg.
Huisartsen worden geconfronteerd met een stijgende werkdruk, toenemende administratieve lasten en complexe organisatorische verwachtingen.
Om de toegankelijkheid van praktijken te monitoren en te verbeteren, maken zorgorganisaties zoals Domus Medica, UGent en eerstelijnszones gebruik van uitgebreide bevragingen en indicatorensets.
Deze evaluaties moeten huisartsen helpen inzicht te krijgen in knelpunten rond onder meer fysieke en organisatorische toegankelijkheid.
In de praktijk blijkt echter dat deze vragenlijsten vaak als tijdrovend, onduidelijk en moeilijk hanteerbaar worden ervaren.
Dit leidt ertoe dat huisartsen dergelijke evaluaties minder frequent uitvoeren, wat de kwaliteit van de dataverzameling en de mogelijkheid tot structurele verbeteringen beperkt.\\\\

Deze bachelorproef vertrekt vanuit een concrete casus binnen Eerstelijnszone Aalst, waar huisartsen aangeven moeite te hebben met het efficiënt en consequent uitvoeren van toegankelijkheidsevaluaties.
De primaire doelgroep bestaat uit huisartsen en praktijkmedewerkers werkzaam binnen deze eerstelijnszone.
Vanuit hun praktijkervaring signaleren zij dat bestaande tools niet afgestemd zijn op de beperkte tijd die beschikbaar is tijdens of naast de consultaties, en dat de interpretatie van bepaalde toegankelijkheidscriteria sterk varieert.\\\\

De centrale onderzoeksvraag van dit onderzoek luidt dan ook:
In welke mate kan een tested proof-of-concept interface bestaande toegankelijkheidsevaluaties voor huisartsen en praktijkmedewerkers sneller, consistenter en gebruiksvriendelijker maken dan de huidige vragenlijsten?\\\\

Om deze hoofdonderzoeksvraag concreet en onderbouwd te beantwoorden, 
wordt ze ondersteund door twee aanvullende subvragen die focussen op de onderliggende mechanismen van verbetering.
Ten eerste wordt onderzocht in welke mate de interface bijdraagt aan een meer consistente interpretatie en beantwoording van toegankelijkheidsindicatoren door huisartsen.
Deze subvraag richt zich op het probleem dat huidige evaluatie-instrumenten vaak ruimte laten voor uiteenlopende interpretaties, wat de betrouwbaarheid van de resultaten vermindert.\\\\

Daarnaast wordt nagegaan in welke mate de interface de cognitieve belasting vermindert tijdens het uitvoeren van een toegankelijkheidsevaluatie in vergelijking met een klassieke vragenlijst.
Hierbij wordt gekeken naar mentale inspanning, navigatiecomplexiteit en momenten van verwarring tijdens het invulproces.
Samen laten deze vragen toe om niet alleen vast te stellen of de interface een verbetering vormt,
maar ook waarom en op welke manier de interface effectiever is dan de huidige aanpak.\\\\

Hoewel toegankelijkheid in de huisartspraktijk een breed en multidimensionaal begrip is,
focust deze bachelorproef bewust op een afgebakende selectie van toegankelijkheidsindicatoren.
Op basis van bestaande literatuur en verkennende interviews met huisartsen wordt een kernset van indicatoren geselecteerd die het vaakst als tijdrovend,
moeilijk interpreteerbaar of inconsistent worden ervaren.
Door deze focus kan de effectiviteit van de proof-of-concept interface op een haalbare en diepgaande manier worden onderzocht.\\\\

Het doel van dit toegepaste onderzoek is daarom tweeledig. 
Enerzijds wordt onderzocht welke elementen van de huidige evaluatie-instrumenten het meeste tijdverlies, verwarring of inconsistentie veroorzaken.
Anderzijds wordt op basis van deze inzichten een proof-of-concept digitale interface ontworpen en getest.
Het beoogde eindresultaat is een werkend prototype dat de toegankelijkheidsevaluatie structureert, visueel ondersteunt en efficiënter maakt, aangevuld met een analyse van de gebruikservaring.
De bachelorproef kan als succesvol beschouwd worden wanneer de interface aantoont dat zij de invultijd reduceert, de gebruiksvriendelijkheid verhoogt en het proces voor huisartsen en praktijkmedewerkers daadwerkelijk vereenvoudigt.
%---------- Stand van zaken ---------------------------------------------------

\section{Literatuurstudie}%
\label{sec:literatuurstudie}

% Hier beschrijf je de \emph{state-of-the-art} rondom je gekozen onderzoeksdomein, d.w.z.\ een inleidende, doorlopende tekst over het onderzoeksdomein van je bachelorproef. Je steunt daarbij heel sterk op de professionele \emph{vakliteratuur}, en niet zozeer op populariserende teksten voor een breed publiek. Wat is de huidige stand van zaken in dit domein, en wat zijn nog eventuele open vragen (die misschien de aanleiding waren tot je onderzoeksvraag!)?

% Je mag de titel van deze sectie ook aanpassen (literatuurstudie, stand van zaken, enz.). Zijn er al gelijkaardige onderzoeken gevoerd? Wat concluderen ze? Wat is het verschil met jouw onderzoek?

% Verwijs bij elke introductie van een term of bewering over het domein naar de vakliteratuur, bijvoorbeeld~\autocite{Hykes2013}! Denk zeker goed na welke werken je refereert en waarom.

% Draag zorg voor correcte literatuurverwijzingen! Een bronvermelding hoort thuis \emph{binnen} de zin waar je je op die bron baseert, dus niet er buiten! Maak meteen een verwijzing als je gebruik maakt van een bron. Doe dit dus \emph{niet} aan het einde van een lange paragraaf. Baseer nooit teveel aansluitende tekst op eenzelfde bron.

% Als je informatie over bronnen verzamelt in JabRef, zorg er dan voor dat alle nodige info aanwezig is om de bron terug te vinden (zoals uitvoerig besproken in de lessen Research Methods).

% Voor literatuurverwijzingen zijn er twee belangrijke commando's:
% \autocite{KEY} => (Auteur, jaartal) Gebruik dit als de naam van de auteur
%   geen onderdeel is van de zin.
% \textcite{KEY} => Auteur (jaartal)  Gebruik dit als de auteursnaam wel een
%   functie heeft in de zin (bv. ``Uit onderzoek door Doll & Hill (1954) bleek
%   ...'')

% Je mag deze sectie nog verder onderverdelen in subsecties als dit de structuur van de tekst kan verduidelijken.
\subsection{Inleiding tot toegankelijkheid in de eerstelijnszorg}
\label{ter:Inleiding tot toegankelijkheid in de eerstelijnszorg}
De toegankelijkheid van huisartsenpraktijken wordt in Vlaanderen gezien als een essentiële pijler van kwaliteitsvolle eerstelijnszorg.
Toegankelijkheid verwijst hierbij naar de mate waarin patiënten tijdig, efficiënt en zonder onnodige drempels een beroep kunnen doen op medische zorg.
\textcite{Rampelberg2025} benadrukt dat toegankelijkheid zowel fysieke als organisatorische dimensies omvat, waarbij factoren zoals infrastructuur, bereikbaarheid, wachttijden en afsprakenbeleid een directe invloed hebben op de toegankelijkheidservaring van patiënten.
Deze brede omschrijving toont aan dat toegankelijkheid een multidimensionaal concept is dat niet eenvoudig te beoordelen is zonder systematische evaluatie.
\\\\
\subsection{Bestaande evaluatie-instrumenten in Vlaanderen}
\label{ter:Bestaande evaluatie-instrumenten in Vlaanderen}
Toegankelijkheidsevaluaties worden in Vlaanderen uitgevoerd met behulp van indicatorensets en vragenlijsten die ontwikkeld zijn door zorgorganisaties zoals Domus Medica en academische partners.
Het rapport Toegankelijkheid Huisartsgeneeskunde identificeert een uitgebreide reeks indicatoren die onder meer betrekking hebben op capaciteit, wachttijden, organisatorische werking en fysieke toegankelijkheid van praktijken.
Hoewel deze indicatoren waardevolle informatie opleveren, wijzen de auteurs erop dat huisartsen deze vragenlijsten vaak als complex en tijdsintensief ervaren door het grote aantal criteria en de nood aan interpretatie bij elke vraag.\autocite{Rampelberg2023}
Dit zorgt ervoor dat evaluaties niet consistent worden uitgevoerd, waardoor zowel de kwaliteit van de data als de bruikbaarheid voor beleidsvorming beperkt blijven.
\\\\
\subsection{Variatie in praktijkorganistaite binnen de eerstelijnszorg}
\label{ter:Variatie in praktijkorganistaite binnen de eerstelijnszorg}
Binnen de Vlaamse eerstelijnszorg zijn grote verschillen zichtbaar in praktijkwerking, schaalgrootte en digitale maturiteit.
Het overlegdocument \textcite{Rampelberg2025a} van Eerstelijnszone Aalst beschrijft het scala aan organisatorische strategieën die praktijken inzetten om de toegankelijkheid te verbeteren, zoals het implementeren van online agendabeheer, samenwerken met naburige praktijken, inzetten van verpleegkundigen of uitbreiden van het zorgaanbod.
Deze diversiteit maakt duidelijk dat een uniforme aanpak voor toegankelijkheidsevaluatie moeilijk toepasbaar is zonder een gebruiksvriendelijke en flexibel inzetbare tool die afgestemd is op verschillende praktijkcontexten.
\\\\
\subsection{Digitale ondersteuning en usability in de zorgsector}
\label{ter:Digitale ondersteuning en usability in de zorgsector}
Internationale literatuur bevestigt dat digitale toepassingen een belangrijke rol kunnen spelen in het vereenvoudigen van administratieve en evaluatieve processen binnen de zorg.
\textcite{Romaric2025} signaleren dat gebruiksvriendelijke interfaces de cognitieve belasting bij zorgverleners verminderen en de consistentie van gegevensinvoer verbeteren wanneer zij zijn afgestemd op de workflow van de gebruiker.
Daarnaast tonen \textcite{Langote2024} aan dat interactieve digitale hulpmiddelen vooral effectief zijn wanneer ze complexe informatie structureren, duidelijke feedback geven tijdens het invullen en visuele hulpmiddelen inzetten om interpretatieproblemen te voorkomen.
Deze inzichten onderbouwen het potentieel van digitale oplossingen voor het stroomlijnen van toegankelijkheidsevaluaties in huisartsenpraktijken.
\\\\
\subsection{Samenvattende tekortkomingen in het huidige veld}
\label{ter:Samenvattende tekortkomingen in het huidige veld}
Ondanks de bestaande evaluatiemethoden en de brede erkenning van het belang van digitale ondersteuning, blijven er significante tekortkomingen in het onderzoeksveld.
Er bestaat momenteel geen gevalideerd digitaal hulpmiddel dat specifiek afgestemd is op de toegankelijkheidsevaluaties van Vlaamse huisartsen.
Daarnaast is weinig bekend over hoe bestaande indicatorensets best vertaald kunnen worden naar een interface die tijdsbesparend is voor huisartsen en praktijkmedewerkers met beperkte administratieve ruimte.
De variatie in interpretatie van indicatoren, zoals beschreven in Vlaamse rapporten, leidt bovendien tot inconsistenties in antwoorden, een probleem waarvan onduidelijk is in welke mate een digitale oplossing dit kan reduceren.
Deze tekortkomingen vormen de aanleiding voor dit onderzoek.
Door het ontwikkelen en testen van een proof-of-concept interface binnen de context van Eerstelijnszone Aalst wil deze bachelorproef bijdragen aan een beter inzicht in hoe digitale hulpmiddelen de toegankelijkheidsevaluatie voor huisartsen kunnen verbeteren.
Het onderzoek sluit daarmee nauw aan bij zowel de Vlaamse beleidscontext als de bestaande internationale vakliteratuur omtrent usability en digitale ondersteuning in de zorg.
%---------- Methodologie ------------------------------------------------------
\section{Methodologie}%
\label{sec:methodologie}

% Hier beschrijf je hoe je van plan bent het onderzoek te voeren. Welke onderzoekstechniek ga je toepassen om elk van je onderzoeksvragen te beantwoorden? Gebruik je hiervoor literatuurstudie, interviews met belanghebbenden (bv.~voor requirements-analyse), experimenten, simulaties, vergelijkende studie, risico-analyse, PoC, \ldots?

% Valt je onderwerp onder één van de typische soorten bachelorproeven die besproken zijn in de lessen Research Methods (bv.\ vergelijkende studie of risico-analyse)? Zorg er dan ook voor dat we duidelijk de verschillende stappen terug vinden die we verwachten in dit soort onderzoek!

% Vermijd onderzoekstechnieken die geen objectieve, meetbare resultaten kunnen opleveren. Enquêtes, bijvoorbeeld, zijn voor een bachelorproef informatica meestal \textbf{niet geschikt}. De antwoorden zijn eerder meningen dan feiten en in de praktijk blijkt het ook bijzonder moeilijk om voldoende respondenten te vinden. Studenten die een enquête willen voeren, hebben meestal ook geen goede definitie van de populatie, waardoor ook niet kan aangetoond worden dat eventuele resultaten representatief zijn.

% Uit dit onderdeel moet duidelijk naar voor komen dat je bachelorproef ook technisch voldoen\-de diepgang zal bevatten. Het zou niet kloppen als een bachelorproef informatica ook door bv.\ een student marketing zou kunnen uitgevoerd worden.

% Je beschrijft ook al welke tools (hardware, software, diensten, \ldots) je denkt hiervoor te gebruiken of te ontwikkelen.

% Probeer ook een tijdschatting te maken. Hoe lang zal je met elke fase van je onderzoek bezig zijn en wat zijn de concrete \emph{deliverables} in elke fase?

Dit onderzoek volgt een Proof-of-Concept-aanpak, een veelgebruikte onderzoeksbenadering binnen toegepaste informatica waarbij een concreet prototype wordt ontwikkeld en geëvalueerd in functie van een reële praktijksituatie.
De centrale vraag is in welke mate een interactieve interface huisartsen en praktijkmedewerkers kan helpen om toegankelijkheidsevaluaties sneller, consistenter en gebruiksvriendelijker uit te voeren.
Om deze vraag te beantwoorden wordt het onderzoek opgedeeld in vier opeenvolgende fasen: een exploratief onderzoek, de ontwikkeling van de interface, een empirische evaluatie en ten slotte de analyse en rapportering van de resultaten.
\subsection{Fase 1}
\label{ter:Fase 1}
De eerste fase bestaat uit een exploratieve analyse, waarin de noden en pijnpunten van huisartsen en huisartsenpraktijken in kaart worden gebracht.
Dit gebeurt via semigestructureerde interviews met een tiental huisartsen of praktijkmedewerkers binnen Eerstelijnszone Aalst.
De keuze voor interviews is bewust: huisartsen en praktijkmedewerkers kunnen tijdens deze gesprekken concreet aangeven welke onderdelen van bestaande toegankelijkheidsbevragingen (zoals de indicatorensets van Domus Medica en UGent) \autocite{Rampelberg2023} als verwarrend, tijdrovend of inefficiënt worden ervaren.
Deze informatie wordt kwalitatief geanalyseerd via thematische codering en vormt de basis voor een set functionele vereisten die de ontwikkeling van de interface sturen.
Tijdens de exploratieve fase wordt geen volledig toegankelijkheidskader gebruikt, maar een gerichte selectie van indicatoren.
Deze selectie gebeurt op basis van een combinatie van literatuuronderzoek en semigestructureerde interviews met huisartsen binnen Eerstelijnszone Aalst.
Enkel de indicatoren die door meerdere bronnen als problematisch worden aangeduid, worden opgenomen in het proof-of-concept.
Naast de interviews wordt een analyse uitgevoerd van bestaande evaluatie-instrumenten, zodat de structuur en inhoud van huidige vragenlijsten correct geïnterpreteerd kunnen worden.
De output van deze fase is een gedetailleerd requirementsdocument dat duidelijk beschrijft welke functies, interactiemechanismen en visuele elementen noodzakelijk zijn om huisartsen optimaal te ondersteunen.
\subsection{Fase 2}
\label{ter:Fase 2}
In de tweede fase wordt op basis van deze inzichten een proof-of-concept interface ontwikkeld.
Dit prototype wordt gerealiseerd met moderne webtechnologieën, waarbij een frontend in React of anguler wordt gecombineerd met een eenvoudige backend.
Voor de gegevensopslag wordt gebruikgemaakt van mockdatabases, zoals JSON-bestanden of een SQLite-database, aangezien de focus van het onderzoek ligt op functionaliteit en gebruikerservaring, niet op een productieklare implementatie.
Het ontwerp van de interface gebeurt iteratief: na de ontwikkeling van vroege prototypes worden deze getest met een tiental gebruikers (bijvoorbeeld huisartsen of praktijkmedewerkers) om snelle feedback te verzamelen.
De interface bevat onder meer een gestroomlijnde vragenstructuur, duidelijke visuele hulpmiddelen, een voortgangsindicator en automatische feedbackmodules.
Deze fase resulteert in een volledig werkend prototype met bijhorende technische documentatie.
\subsection{Fase 3}
\label{ter:Fase 3}
De derde fase omvat de empirische evaluatie, waarin systematisch wordt onderzocht in welke mate de interface het evaluatieproces daadwerkelijk verbetert.
Hierbij worden een tiental huisartsen en praktijkmedewerkers betrokken.
Om objectieve, meetbare resultaten te bekomen, wordt een combinatie van kwantitatieve en kwalitatieve technieken gebruikt.
Ten eerste wordt onderzocht of de interface het proces sneller maakt: elke deelnemer doorloopt zowel een klassieke toegankelijkheidsvragenlijst als het digitale prototype, waarbij de invultijd nauwkeurig wordt geregistreerd.
Vervolgens wordt de gebruiksvriendelijkheid beoordeeld via de bekende System Usability Scale (SUS), een gestandaardiseerde methode die toelaat de interface te situeren ten opzichte van internationaal erkende usabilitynormen.
Daarnaast wordt de think-aloud-methode toegepast: tijdens het invullen spreekt de gebruiker zijn gedachten uit, wat waardevolle inzichten oplevert over interpretatieproblemen, verwarring en cognitieve belasting.
De interactie met de interface wordt verder gemonitord via heatmaps en klikgegevens, verzameld met tools zoals Microsoft Clarity of Hotjar.
Deze registreren onder meer klikgedrag, scrollgedrag, zones van verwarring (zoals rage clicks) en time-on-task.
Tot slot wordt een consistency-check uitgevoerd: dezelfde indicatoren worden via de klassieke vragenlijst en de interface afgenomen, zodat kan worden nagegaan of het prototype leidt tot minder variatie of foutieve interpretaties.
\subsection{Fase 4}
\label{ter:Fase 4}
In de vierde fase worden de verzamelde gegevens grondig geanalyseerd en geïnterpreteerd.
De kwantitatieve data, zoals invultijden, SUS-scores en heatmapstatistieken, worden verwerkt in Excel of Python om verschillen tussen de klassieke methode en de PoC-interface te bepalen.
De kwalitatieve data uit interviews, think-aloud-sessies en observaties worden thematisch gecodeerd en vergeleken met de bevindingen uit literatuur.
De combinatie van beide datatypen maakt het mogelijk om een genuanceerd antwoord te geven op de onderzoeksvraag en om te beoordelen welke onderdelen van de interface effectief bijdragen aan efficiëntie en gebruiksvriendelijkheid, en welke nog verdere verfijning vereisen.

\subsection{Resultaat}
\label{ter:Resultaat}
Tot slot worden alle resultaten samengebracht in een eindrapport dat naast de empirische bevindingen ook aanbevelingen bevat voor verdere ontwikkeling of implementatie van de interface in een bredere eerstelijnscontext.
De totale doorlooptijd van het onderzoek bedraagt ongeveer twaalf tot vijftien weken, waarbij de exploratie en requirementsanalyse twee à drie weken in beslag nemen, de ontwikkeling van het prototype vijf à zes weken, de evaluatiefase drie à vier weken en de analyse en rapportering twee weken.
De bachelorproef wordt als succesvol beschouwd wanneer aangetoond kan worden dat het prototype het evaluatiemoment voor huisartsen efficiënter maakt, gebruiksvriendelijker vormgeeft en de consistentie van de resultaten verhoogt.

%---------- Verwachte resultaten ----------------------------------------------
\section{Verwacht resultaat, conclusie}%
\label{sec:verwachte_resultaten}

% Hier beschrijf je welke resultaten je verwacht. Als je metingen en simulaties uitvoert, kan je hier al mock-ups maken van de grafieken samen met de verwachte conclusies. Benoem zeker al je assen en de onderdelen van de grafiek die je gaat gebruiken. Dit zorgt ervoor dat je concreet weet welk soort data je moet verzamelen en hoe je die moet meten.

% Wat heeft de doelgroep van je onderzoek aan het resultaat? Op welke manier zorgt jouw bachelorproef voor een meerwaarde?

% Hier beschrijf je wat je verwacht uit je onderzoek, met de motivatie waarom. Het is \textbf{niet} erg indien uit je onderzoek andere resultaten en conclusies vloeien dan dat je hier beschrijft: het is dan juist interessant om te onderzoeken waarom jouw hypothesen niet overeenkomen met de resultaten.
Op basis van de literatuur en de gekende knelpunten in de huidige toegankelijkheidsevaluaties wordt verwacht dat het proof-of-concept een duidelijke verbetering zal bieden op vlak van efficiëntie, gebruiksvriendelijkheid en consistentie.
De interface is ontworpen om huisartsen en praktijkmedewerkers stapsgewijs te begeleiden en cognitieve belasting te verminderen, waardoor de totale invultijd naar verwachting aanzienlijk lager zal liggen dan bij de klassieke vragenlijsten.
Grafisch vertaald betekent dit dat in een eenvoudige tijdsvergelijking de balk die de invultijd van de interface vertegenwoordigt, duidelijk onder die van de traditionele methode zal liggen.
Ook voor gebruiksvriendelijkheid wordt een positieve uitkomst verwacht: op basis van de System Usability Scale zou de interface een score behalen die boven de drempel van 75 ligt, wat internationaal beschouwd wordt als “good usability”.
Daarnaast wordt verwacht dat de interface de consistentie van antwoorden verhoogt, omdat zij definities, voorbeelden en visuele ondersteuning biedt bij complexere indicatoren.
Een vergelijking van overeenkomende antwoorden tussen klassieke en digitale invulling zou dus een hoger consistentiepercentage moeten tonen bij het prototype. Heatmap-analyses (via Clarity of Hotjar) zullen waarschijnlijk eveneens aantonen dat de interface minder zones van verwarring veroorzaakt, met minder ongewenste kliks en vloeiender navigatiegedrag dan in de klassieke evaluaties.
Voor de doelgroep, huisartsen en huisartsenpraktijken binnen de Eerstelijnszone Aalst, betekent dit dat toegankelijkheidsevaluaties minder tijd en moeite zullen kosten, waardoor ze realistischer worden om regelmatig uit te voeren.
De verwachte verbeteringen dragen bij aan betrouwbaardere toegankelijkheidsdata, minder administratieve belasting en een duidelijker zicht op verbeterkansen in de praktijk.
Tegelijk blijven deze verwachtingen hypothesen: het onderzoek staat open voor afwijkende resultaten, die op hun beurt waardevolle inzichten kunnen opleveren over waarom bepaalde ontwerpkeuzes al dan niet functioneren in de reële zorgcontext.


%%---------- Andere bijlagen --------------------------------------------------
% TODO: Voeg hier eventuele andere bijlagen toe. Bv. als je deze BP voor de
% tweede keer indient, een overzicht van de verbeteringen t.o.v. het origineel.
%\input{...}

%%---------- Backmatter, referentielijst ---------------------------------------

\backmatter{}

\setlength\bibitemsep{2pt} %% Add Some space between the bibliograpy entries
\printbibliography[heading=bibintoc]

\end{document}
