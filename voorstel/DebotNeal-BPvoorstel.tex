%==============================================================================
% Sjabloon onderzoeksvoorstel bachproef
%==============================================================================
% Gebaseerd op document class `hogent-article'
% zie <https://github.com/HoGentTIN/latex-hogent-article>

% Voor een voorstel in het Engels: voeg de documentclass-optie [english] toe.
% Let op: kan enkel na toestemming van de bachelorproefcoördinator!
\documentclass{hogent-article}

% Invoegen bibliografiebestand
\addbibresource{voorstel.bib}

% Informatie over de opleiding, het vak en soort opdracht
\studyprogramme{Professionele bachelor toegepaste informatica}
\course{Bachelorproef}
\assignmenttype{Onderzoeksvoorstel}
% Voor een voorstel in het Engels, haal de volgende 3 regels uit commentaar
% \studyprogramme{Bachelor of applied information technology}
% \course{Bachelor thesis}
% \assignmenttype{Research proposal}

\academicyear{2025-2026} % TODO: pas het academiejaar aan

% TODO: Werktitel
\title{Ontwikkeling en evaluatie van een proof-of-concept interface voor het efficiënt beoordelen van de toegankelijkheid van Vlaamse huisartsenpraktijken}

% TODO: Studentnaam en emailadres invullen
\author{Neal Debot}
\email{neal.debot@student.hogent.be}

% TODO: Medestudent
% Gaat het om een bachelorproef in samenwerking met een student in een andere
% opleiding? Geef dan de naam en emailadres hier
% \author{Yasmine Alaoui (naam opleiding)}
% \email{yasmine.alaoui@student.hogent.be}

\projectrepo{https://github.com/NealDebot/BachelorProef-NealDebot}

% TODO: Geef de co-promotor op
\supervisor[Co-promotor]{S. Van Rampelberg (\href{mailto:sabien.vanrampelberg@odisee.be}{sabien.vanrampelberg@odisee.be})}

% Binnen welke specialisatierichting uit 3TI situeert dit onderzoek zich?
% Kies uit deze lijst:
%
% - Mobile \& Enterprise development
% - AI \& Data Engineering
% - Functional \& Business Analysis
% - System \& Network Administrator
% - Mainframe Expert
% - Als het onderzoek niet past binnen een van deze domeinen specifieer je deze
%   zelf
%
\specialisation{Mobile \& Enterprise development}
\keywords{Toegankelijkheid huisartsenpraktijken, Usability testing, Proof-of-concept}

\begin{document}

\begin{abstract}
  De toegankelijkheid van huisartsenpraktijken wordt in Vlaanderen regelmatig geëvalueerd via indicatorensets en vragenlijsten ontwikkeld door o.a. Domus Medica en UGent.
  Uit eerdere pilootstudies blijkt dat deze evaluaties door huisartsen vaak als tijdrovend, complex en inconsistent worden ervaren, wat de frequentie en kwaliteit van zelfevaluaties beperkt.
  Dit onderzoek zal zich toespitsen op de toegankelijkheid met betreking tot de fysieke en organisatorische aspecten van de praktijk.
  En zal onderzoeken in welke mate een proof-of-concept interface dit proces efficiënter en gebruiksvriendelijker kan maken.\\\\
  Via interviews met huisartsen wordt eerst in kaart gebracht welke elementen van bestaande evaluaties de grootste tijdsdruk of verwarring veroorzaken.
  Op basis hiervan wordt een interactieve interface ontwikkeld die de toegankelijkheidsbevraging structureert, visuele ondersteuning biedt en automatische feedback genereert.
  De interface wordt vervolgens getest bij huisartsen aan de hand van tijdsmetingen, usability-observaties, Microsoft Clarity-heatmaps en de System Usability Scale (SUS).\\\\
  De resultaten geven inzicht in de mate waarin de interface het invulproces versnelt, de consistentie van antwoorden verhoogt, sneller feedback geeft en de gebruikerservaring verbetert.
  Het onderzoek toont aan hoe digitale ondersteuning het evalueren van toegankelijkheid kan vereenvoudigen en vormt een basis voor verdere implementatie binnen de eerstelijnszorg.
\end{abstract}

\tableofcontents

% De hoofdtekst van het voorstel zit in een apart bestand, zodat het makkelijk
% kan opgenomen worden in de bijlagen van de bachelorproef zelf.
%---------- Inleiding ---------------------------------------------------------

% TODO: Is dit voorstel gebaseerd op een paper van Research Methods die je
% vorig jaar hebt ingediend? Heb je daarbij eventueel samengewerkt met een
% andere student?
% Zo ja, haal dan de tekst hieronder uit commentaar en pas aan.

%\paragraph{Opmerking}

% Dit voorstel is gebaseerd op het onderzoeksvoorstel dat werd geschreven in het
% kader van het vak Research Methods dat ik (vorig/dit) academiejaar heb
% uitgewerkt (met medesturent VOORNAAM NAAM als mede-auteur).
% 

\section{Inleiding}%
\label{sec:inleiding}

% Waarover zal je bachelorproef gaan? Introduceer het thema en zorg dat volgende zaken zeker duidelijk aanwezig zijn:

% \begin{itemize}
%   \item kaderen thema
%   \item de doelgroep
%   \item de probleemstelling en (centrale) onderzoeksvraag
%   \item de onderzoeksdoelstelling
% \end{itemize}

% Denk er aan: een typische bachelorproef is \textit{toegepast onderzoek}, wat betekent dat je start vanuit een concrete probleemsituatie in bedrijfscontext, een \textbf{casus}. Het is belangrijk om je onderwerp goed af te bakenen: je gaat voor die \textit{ene specifieke probleemsituatie} op zoek naar een goede oplossing, op basis van de huidige kennis in het vakgebied.

% De doelgroep moet ook concreet en duidelijk zijn, dus geen algemene of vaag gedefinieerde groepen zoals \emph{bedrijven}, \emph{developers}, \emph{Vlamingen}, enz. Je richt je in elk geval op it-professionals, een bachelorproef is geen populariserende tekst. Eén specifiek bedrijf (die te maken hebben met een concrete probleemsituatie) is dus beter dan \emph{bedrijven} in het algemeen.

% Formuleer duidelijk de onderzoeksvraag! De begeleiders lezen nog steeds te veel voorstellen waarin we geen onderzoeksvraag terugvinden.

% Schrijf ook iets over de doelstelling. Wat zie je als het concrete eindresultaat van je onderzoek, naast de uitgeschreven scriptie? Is het een proof-of-concept, een rapport met aanbevelingen, \ldots Met welk eindresultaat kan je je bachelorproef als een succes beschouwen?
De toegankelijkheid van huisartsenpraktijken vormt een groeiende uitdaging binnen de Vlaamse eerstelijnszorg.
Huisartsen worden geconfronteerd met een stijgende werkdruk, toenemende administratieve lasten en complexe organisatorische verwachtingen.
Om de toegankelijkheid van praktijken te monitoren en te verbeteren, maken zorgorganisaties zoals Domus Medica, UGent en eerstelijnszones gebruik van uitgebreide bevragingen en indicatorensets.
Deze evaluaties moeten huisartsen helpen inzicht te krijgen in knelpunten rond onder meer fysieke en organisatorische toegankelijkheid.
In de praktijk blijkt echter dat deze vragenlijsten vaak als tijdrovend, onduidelijk en moeilijk hanteerbaar worden ervaren.
Dit leidt ertoe dat huisartsen dergelijke evaluaties minder frequent uitvoeren, wat de kwaliteit van de dataverzameling en de mogelijkheid tot structurele verbeteringen beperkt.\\\\

Deze bachelorproef vertrekt vanuit een concrete casus binnen Eerstelijnszone Aalst, waar huisartsen aangeven moeite te hebben met het efficiënt en consequent uitvoeren van toegankelijkheidsevaluaties.
De primaire doelgroep bestaat uit huisartsen en praktijkmedewerkers werkzaam binnen deze eerstelijnszone.
Vanuit hun praktijkervaring signaleren zij dat bestaande tools niet afgestemd zijn op de beperkte tijd die beschikbaar is tijdens of naast de consultaties, en dat de interpretatie van bepaalde toegankelijkheidscriteria sterk varieert.\\\\

De centrale onderzoeksvraag van dit onderzoek luidt dan ook:
In welke mate kan een tested proof-of-concept interface bestaande toegankelijkheidsevaluaties voor huisartsen en praktijkmedewerkers sneller, consistenter en gebruiksvriendelijker maken dan de huidige vragenlijsten?\\\\

Om deze hoofdonderzoeksvraag concreet en onderbouwd te beantwoorden, 
wordt ze ondersteund door twee aanvullende subvragen die focussen op de onderliggende mechanismen van verbetering.
Ten eerste wordt onderzocht in welke mate de interface bijdraagt aan een meer consistente interpretatie en beantwoording van toegankelijkheidsindicatoren door huisartsen.
Deze subvraag richt zich op het probleem dat huidige evaluatie-instrumenten vaak ruimte laten voor uiteenlopende interpretaties, wat de betrouwbaarheid van de resultaten vermindert.\\\\

Daarnaast wordt nagegaan in welke mate de interface de cognitieve belasting vermindert tijdens het uitvoeren van een toegankelijkheidsevaluatie in vergelijking met een klassieke vragenlijst.
Hierbij wordt gekeken naar mentale inspanning, navigatiecomplexiteit en momenten van verwarring tijdens het invulproces.
Samen laten deze vragen toe om niet alleen vast te stellen of de interface een verbetering vormt,
maar ook waarom en op welke manier de interface effectiever is dan de huidige aanpak.\\\\

Hoewel toegankelijkheid in de huisartspraktijk een breed en multidimensionaal begrip is,
focust deze bachelorproef bewust op een afgebakende selectie van toegankelijkheidsindicatoren.
Op basis van bestaande literatuur en verkennende interviews met huisartsen wordt een kernset van indicatoren geselecteerd die het vaakst als tijdrovend,
moeilijk interpreteerbaar of inconsistent worden ervaren.
Door deze focus kan de effectiviteit van de proof-of-concept interface op een haalbare en diepgaande manier worden onderzocht.\\\\

Het doel van dit toegepaste onderzoek is daarom tweeledig. 
Enerzijds wordt onderzocht welke elementen van de huidige evaluatie-instrumenten het meeste tijdverlies, verwarring of inconsistentie veroorzaken.
Anderzijds wordt op basis van deze inzichten een proof-of-concept digitale interface ontworpen en getest.
Het beoogde eindresultaat is een werkend prototype dat de toegankelijkheidsevaluatie structureert, visueel ondersteunt en efficiënter maakt, aangevuld met een analyse van de gebruikservaring.
De bachelorproef kan als succesvol beschouwd worden wanneer de interface aantoont dat zij de invultijd reduceert, de gebruiksvriendelijkheid verhoogt en het proces voor huisartsen en praktijkmedewerkers daadwerkelijk vereenvoudigt.
%---------- Stand van zaken ---------------------------------------------------

\section{Literatuurstudie}%
\label{sec:literatuurstudie}

% Hier beschrijf je de \emph{state-of-the-art} rondom je gekozen onderzoeksdomein, d.w.z.\ een inleidende, doorlopende tekst over het onderzoeksdomein van je bachelorproef. Je steunt daarbij heel sterk op de professionele \emph{vakliteratuur}, en niet zozeer op populariserende teksten voor een breed publiek. Wat is de huidige stand van zaken in dit domein, en wat zijn nog eventuele open vragen (die misschien de aanleiding waren tot je onderzoeksvraag!)?

% Je mag de titel van deze sectie ook aanpassen (literatuurstudie, stand van zaken, enz.). Zijn er al gelijkaardige onderzoeken gevoerd? Wat concluderen ze? Wat is het verschil met jouw onderzoek?

% Verwijs bij elke introductie van een term of bewering over het domein naar de vakliteratuur, bijvoorbeeld~\autocite{Hykes2013}! Denk zeker goed na welke werken je refereert en waarom.

% Draag zorg voor correcte literatuurverwijzingen! Een bronvermelding hoort thuis \emph{binnen} de zin waar je je op die bron baseert, dus niet er buiten! Maak meteen een verwijzing als je gebruik maakt van een bron. Doe dit dus \emph{niet} aan het einde van een lange paragraaf. Baseer nooit teveel aansluitende tekst op eenzelfde bron.

% Als je informatie over bronnen verzamelt in JabRef, zorg er dan voor dat alle nodige info aanwezig is om de bron terug te vinden (zoals uitvoerig besproken in de lessen Research Methods).

% Voor literatuurverwijzingen zijn er twee belangrijke commando's:
% \autocite{KEY} => (Auteur, jaartal) Gebruik dit als de naam van de auteur
%   geen onderdeel is van de zin.
% \textcite{KEY} => Auteur (jaartal)  Gebruik dit als de auteursnaam wel een
%   functie heeft in de zin (bv. ``Uit onderzoek door Doll & Hill (1954) bleek
%   ...'')

% Je mag deze sectie nog verder onderverdelen in subsecties als dit de structuur van de tekst kan verduidelijken.
\subsection{Inleiding tot toegankelijkheid in de eerstelijnszorg}
\label{ter:Inleiding tot toegankelijkheid in de eerstelijnszorg}
De toegankelijkheid van huisartsenpraktijken wordt in Vlaanderen gezien als een essentiële pijler van kwaliteitsvolle eerstelijnszorg.
Toegankelijkheid verwijst hierbij naar de mate waarin patiënten tijdig, efficiënt en zonder onnodige drempels een beroep kunnen doen op medische zorg.
\textcite{Rampelberg2025} benadrukt dat toegankelijkheid zowel fysieke als organisatorische dimensies omvat, waarbij factoren zoals infrastructuur, bereikbaarheid, wachttijden en afsprakenbeleid een directe invloed hebben op de toegankelijkheidservaring van patiënten.
Deze brede omschrijving toont aan dat toegankelijkheid een multidimensionaal concept is dat niet eenvoudig te beoordelen is zonder systematische evaluatie.
\\\\
\subsection{Bestaande evaluatie-instrumenten in Vlaanderen}
\label{ter:Bestaande evaluatie-instrumenten in Vlaanderen}
Toegankelijkheidsevaluaties worden in Vlaanderen uitgevoerd met behulp van indicatorensets en vragenlijsten die ontwikkeld zijn door zorgorganisaties zoals Domus Medica en academische partners.
Het rapport Toegankelijkheid Huisartsgeneeskunde identificeert een uitgebreide reeks indicatoren die onder meer betrekking hebben op capaciteit, wachttijden, organisatorische werking en fysieke toegankelijkheid van praktijken.
Hoewel deze indicatoren waardevolle informatie opleveren, wijzen de auteurs erop dat huisartsen deze vragenlijsten vaak als complex en tijdsintensief ervaren door het grote aantal criteria en de nood aan interpretatie bij elke vraag.\autocite{Rampelberg2023}
Dit zorgt ervoor dat evaluaties niet consistent worden uitgevoerd, waardoor zowel de kwaliteit van de data als de bruikbaarheid voor beleidsvorming beperkt blijven.
\\\\
\subsection{Variatie in praktijkorganistaite binnen de eerstelijnszorg}
\label{ter:Variatie in praktijkorganistaite binnen de eerstelijnszorg}
Binnen de Vlaamse eerstelijnszorg zijn grote verschillen zichtbaar in praktijkwerking, schaalgrootte en digitale maturiteit.
Het overlegdocument \textcite{Rampelberg2025a} van Eerstelijnszone Aalst beschrijft het scala aan organisatorische strategieën die praktijken inzetten om de toegankelijkheid te verbeteren, zoals het implementeren van online agendabeheer, samenwerken met naburige praktijken, inzetten van verpleegkundigen of uitbreiden van het zorgaanbod.
Deze diversiteit maakt duidelijk dat een uniforme aanpak voor toegankelijkheidsevaluatie moeilijk toepasbaar is zonder een gebruiksvriendelijke en flexibel inzetbare tool die afgestemd is op verschillende praktijkcontexten.
\\\\
\subsection{Digitale ondersteuning en usability in de zorgsector}
\label{ter:Digitale ondersteuning en usability in de zorgsector}
Internationale literatuur bevestigt dat digitale toepassingen een belangrijke rol kunnen spelen in het vereenvoudigen van administratieve en evaluatieve processen binnen de zorg.
\textcite{Romaric2025} signaleren dat gebruiksvriendelijke interfaces de cognitieve belasting bij zorgverleners verminderen en de consistentie van gegevensinvoer verbeteren wanneer zij zijn afgestemd op de workflow van de gebruiker.
Daarnaast tonen \textcite{Langote2024} aan dat interactieve digitale hulpmiddelen vooral effectief zijn wanneer ze complexe informatie structureren, duidelijke feedback geven tijdens het invullen en visuele hulpmiddelen inzetten om interpretatieproblemen te voorkomen.
Deze inzichten onderbouwen het potentieel van digitale oplossingen voor het stroomlijnen van toegankelijkheidsevaluaties in huisartsenpraktijken.
\\\\
\subsection{Samenvattende tekortkomingen in het huidige veld}
\label{ter:Samenvattende tekortkomingen in het huidige veld}
Ondanks de bestaande evaluatiemethoden en de brede erkenning van het belang van digitale ondersteuning, blijven er significante tekortkomingen in het onderzoeksveld.
Er bestaat momenteel geen gevalideerd digitaal hulpmiddel dat specifiek afgestemd is op de toegankelijkheidsevaluaties van Vlaamse huisartsen.
Daarnaast is weinig bekend over hoe bestaande indicatorensets best vertaald kunnen worden naar een interface die tijdsbesparend is voor huisartsen en praktijkmedewerkers met beperkte administratieve ruimte.
De variatie in interpretatie van indicatoren, zoals beschreven in Vlaamse rapporten, leidt bovendien tot inconsistenties in antwoorden, een probleem waarvan onduidelijk is in welke mate een digitale oplossing dit kan reduceren.
Deze tekortkomingen vormen de aanleiding voor dit onderzoek.
Door het ontwikkelen en testen van een proof-of-concept interface binnen de context van Eerstelijnszone Aalst wil deze bachelorproef bijdragen aan een beter inzicht in hoe digitale hulpmiddelen de toegankelijkheidsevaluatie voor huisartsen kunnen verbeteren.
Het onderzoek sluit daarmee nauw aan bij zowel de Vlaamse beleidscontext als de bestaande internationale vakliteratuur omtrent usability en digitale ondersteuning in de zorg.
%---------- Methodologie ------------------------------------------------------
\section{Methodologie}%
\label{sec:methodologie}

% Hier beschrijf je hoe je van plan bent het onderzoek te voeren. Welke onderzoekstechniek ga je toepassen om elk van je onderzoeksvragen te beantwoorden? Gebruik je hiervoor literatuurstudie, interviews met belanghebbenden (bv.~voor requirements-analyse), experimenten, simulaties, vergelijkende studie, risico-analyse, PoC, \ldots?

% Valt je onderwerp onder één van de typische soorten bachelorproeven die besproken zijn in de lessen Research Methods (bv.\ vergelijkende studie of risico-analyse)? Zorg er dan ook voor dat we duidelijk de verschillende stappen terug vinden die we verwachten in dit soort onderzoek!

% Vermijd onderzoekstechnieken die geen objectieve, meetbare resultaten kunnen opleveren. Enquêtes, bijvoorbeeld, zijn voor een bachelorproef informatica meestal \textbf{niet geschikt}. De antwoorden zijn eerder meningen dan feiten en in de praktijk blijkt het ook bijzonder moeilijk om voldoende respondenten te vinden. Studenten die een enquête willen voeren, hebben meestal ook geen goede definitie van de populatie, waardoor ook niet kan aangetoond worden dat eventuele resultaten representatief zijn.

% Uit dit onderdeel moet duidelijk naar voor komen dat je bachelorproef ook technisch voldoen\-de diepgang zal bevatten. Het zou niet kloppen als een bachelorproef informatica ook door bv.\ een student marketing zou kunnen uitgevoerd worden.

% Je beschrijft ook al welke tools (hardware, software, diensten, \ldots) je denkt hiervoor te gebruiken of te ontwikkelen.

% Probeer ook een tijdschatting te maken. Hoe lang zal je met elke fase van je onderzoek bezig zijn en wat zijn de concrete \emph{deliverables} in elke fase?

Dit onderzoek volgt een Proof-of-Concept-aanpak, een veelgebruikte onderzoeksbenadering binnen toegepaste informatica waarbij een concreet prototype wordt ontwikkeld en geëvalueerd in functie van een reële praktijksituatie.
De centrale vraag is in welke mate een interactieve interface huisartsen en praktijkmedewerkers kan helpen om toegankelijkheidsevaluaties sneller, consistenter en gebruiksvriendelijker uit te voeren.
Om deze vraag te beantwoorden wordt het onderzoek opgedeeld in vier opeenvolgende fasen: een exploratief onderzoek, de ontwikkeling van de interface, een empirische evaluatie en ten slotte de analyse en rapportering van de resultaten.
\subsection{Fase 1}
\label{ter:Fase 1}
De eerste fase bestaat uit een exploratieve analyse, waarin de noden en pijnpunten van huisartsen en huisartsenpraktijken in kaart worden gebracht.
Dit gebeurt via semigestructureerde interviews met een tiental huisartsen of praktijkmedewerkers binnen Eerstelijnszone Aalst.
De keuze voor interviews is bewust: huisartsen en praktijkmedewerkers kunnen tijdens deze gesprekken concreet aangeven welke onderdelen van bestaande toegankelijkheidsbevragingen (zoals de indicatorensets van Domus Medica en UGent) \autocite{Rampelberg2023} als verwarrend, tijdrovend of inefficiënt worden ervaren.
Deze informatie wordt kwalitatief geanalyseerd via thematische codering en vormt de basis voor een set functionele vereisten die de ontwikkeling van de interface sturen.
Tijdens de exploratieve fase wordt geen volledig toegankelijkheidskader gebruikt, maar een gerichte selectie van indicatoren.
Deze selectie gebeurt op basis van een combinatie van literatuuronderzoek en semigestructureerde interviews met huisartsen binnen Eerstelijnszone Aalst.
Enkel de indicatoren die door meerdere bronnen als problematisch worden aangeduid, worden opgenomen in het proof-of-concept.
Naast de interviews wordt een analyse uitgevoerd van bestaande evaluatie-instrumenten, zodat de structuur en inhoud van huidige vragenlijsten correct geïnterpreteerd kunnen worden.
De output van deze fase is een gedetailleerd requirementsdocument dat duidelijk beschrijft welke functies, interactiemechanismen en visuele elementen noodzakelijk zijn om huisartsen optimaal te ondersteunen.
\subsection{Fase 2}
\label{ter:Fase 2}
In de tweede fase wordt op basis van deze inzichten een proof-of-concept interface ontwikkeld.
Dit prototype wordt gerealiseerd met moderne webtechnologieën, waarbij een frontend in React of anguler wordt gecombineerd met een eenvoudige backend.
Voor de gegevensopslag wordt gebruikgemaakt van mockdatabases, zoals JSON-bestanden of een SQLite-database, aangezien de focus van het onderzoek ligt op functionaliteit en gebruikerservaring, niet op een productieklare implementatie.
Het ontwerp van de interface gebeurt iteratief: na de ontwikkeling van vroege prototypes worden deze getest met een tiental gebruikers (bijvoorbeeld huisartsen of praktijkmedewerkers) om snelle feedback te verzamelen.
De interface bevat onder meer een gestroomlijnde vragenstructuur, duidelijke visuele hulpmiddelen, een voortgangsindicator en automatische feedbackmodules.
Deze fase resulteert in een volledig werkend prototype met bijhorende technische documentatie.
\subsection{Fase 3}
\label{ter:Fase 3}
De derde fase omvat de empirische evaluatie, waarin systematisch wordt onderzocht in welke mate de interface het evaluatieproces daadwerkelijk verbetert.
Hierbij worden een tiental huisartsen en praktijkmedewerkers betrokken.
Om objectieve, meetbare resultaten te bekomen, wordt een combinatie van kwantitatieve en kwalitatieve technieken gebruikt.
Ten eerste wordt onderzocht of de interface het proces sneller maakt: elke deelnemer doorloopt zowel een klassieke toegankelijkheidsvragenlijst als het digitale prototype, waarbij de invultijd nauwkeurig wordt geregistreerd.
Vervolgens wordt de gebruiksvriendelijkheid beoordeeld via de bekende System Usability Scale (SUS), een gestandaardiseerde methode die toelaat de interface te situeren ten opzichte van internationaal erkende usabilitynormen.
Daarnaast wordt de think-aloud-methode toegepast: tijdens het invullen spreekt de gebruiker zijn gedachten uit, wat waardevolle inzichten oplevert over interpretatieproblemen, verwarring en cognitieve belasting.
De interactie met de interface wordt verder gemonitord via heatmaps en klikgegevens, verzameld met tools zoals Microsoft Clarity of Hotjar.
Deze registreren onder meer klikgedrag, scrollgedrag, zones van verwarring (zoals rage clicks) en time-on-task.
Tot slot wordt een consistency-check uitgevoerd: dezelfde indicatoren worden via de klassieke vragenlijst en de interface afgenomen, zodat kan worden nagegaan of het prototype leidt tot minder variatie of foutieve interpretaties.
\subsection{Fase 4}
\label{ter:Fase 4}
In de vierde fase worden de verzamelde gegevens grondig geanalyseerd en geïnterpreteerd.
De kwantitatieve data, zoals invultijden, SUS-scores en heatmapstatistieken, worden verwerkt in Excel of Python om verschillen tussen de klassieke methode en de PoC-interface te bepalen.
De kwalitatieve data uit interviews, think-aloud-sessies en observaties worden thematisch gecodeerd en vergeleken met de bevindingen uit literatuur.
De combinatie van beide datatypen maakt het mogelijk om een genuanceerd antwoord te geven op de onderzoeksvraag en om te beoordelen welke onderdelen van de interface effectief bijdragen aan efficiëntie en gebruiksvriendelijkheid, en welke nog verdere verfijning vereisen.

\subsection{Resultaat}
\label{ter:Resultaat}
Tot slot worden alle resultaten samengebracht in een eindrapport dat naast de empirische bevindingen ook aanbevelingen bevat voor verdere ontwikkeling of implementatie van de interface in een bredere eerstelijnscontext.
De totale doorlooptijd van het onderzoek bedraagt ongeveer twaalf tot vijftien weken, waarbij de exploratie en requirementsanalyse twee à drie weken in beslag nemen, de ontwikkeling van het prototype vijf à zes weken, de evaluatiefase drie à vier weken en de analyse en rapportering twee weken.
De bachelorproef wordt als succesvol beschouwd wanneer aangetoond kan worden dat het prototype het evaluatiemoment voor huisartsen efficiënter maakt, gebruiksvriendelijker vormgeeft en de consistentie van de resultaten verhoogt.

%---------- Verwachte resultaten ----------------------------------------------
\section{Verwacht resultaat, conclusie}%
\label{sec:verwachte_resultaten}

% Hier beschrijf je welke resultaten je verwacht. Als je metingen en simulaties uitvoert, kan je hier al mock-ups maken van de grafieken samen met de verwachte conclusies. Benoem zeker al je assen en de onderdelen van de grafiek die je gaat gebruiken. Dit zorgt ervoor dat je concreet weet welk soort data je moet verzamelen en hoe je die moet meten.

% Wat heeft de doelgroep van je onderzoek aan het resultaat? Op welke manier zorgt jouw bachelorproef voor een meerwaarde?

% Hier beschrijf je wat je verwacht uit je onderzoek, met de motivatie waarom. Het is \textbf{niet} erg indien uit je onderzoek andere resultaten en conclusies vloeien dan dat je hier beschrijft: het is dan juist interessant om te onderzoeken waarom jouw hypothesen niet overeenkomen met de resultaten.
Op basis van de literatuur en de gekende knelpunten in de huidige toegankelijkheidsevaluaties wordt verwacht dat het proof-of-concept een duidelijke verbetering zal bieden op vlak van efficiëntie, gebruiksvriendelijkheid en consistentie.
De interface is ontworpen om huisartsen en praktijkmedewerkers stapsgewijs te begeleiden en cognitieve belasting te verminderen, waardoor de totale invultijd naar verwachting aanzienlijk lager zal liggen dan bij de klassieke vragenlijsten.
Grafisch vertaald betekent dit dat in een eenvoudige tijdsvergelijking de balk die de invultijd van de interface vertegenwoordigt, duidelijk onder die van de traditionele methode zal liggen.
Ook voor gebruiksvriendelijkheid wordt een positieve uitkomst verwacht: op basis van de System Usability Scale zou de interface een score behalen die boven de drempel van 75 ligt, wat internationaal beschouwd wordt als “good usability”.
Daarnaast wordt verwacht dat de interface de consistentie van antwoorden verhoogt, omdat zij definities, voorbeelden en visuele ondersteuning biedt bij complexere indicatoren.
Een vergelijking van overeenkomende antwoorden tussen klassieke en digitale invulling zou dus een hoger consistentiepercentage moeten tonen bij het prototype. Heatmap-analyses (via Clarity of Hotjar) zullen waarschijnlijk eveneens aantonen dat de interface minder zones van verwarring veroorzaakt, met minder ongewenste kliks en vloeiender navigatiegedrag dan in de klassieke evaluaties.
Voor de doelgroep, huisartsen en huisartsenpraktijken binnen de Eerstelijnszone Aalst, betekent dit dat toegankelijkheidsevaluaties minder tijd en moeite zullen kosten, waardoor ze realistischer worden om regelmatig uit te voeren.
De verwachte verbeteringen dragen bij aan betrouwbaardere toegankelijkheidsdata, minder administratieve belasting en een duidelijker zicht op verbeterkansen in de praktijk.
Tegelijk blijven deze verwachtingen hypothesen: het onderzoek staat open voor afwijkende resultaten, die op hun beurt waardevolle inzichten kunnen opleveren over waarom bepaalde ontwerpkeuzes al dan niet functioneren in de reële zorgcontext.


\printbibliography[heading=bibintoc]

\end{document}